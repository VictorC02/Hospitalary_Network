%!TeX root=MemoriaTFG.tex

\chapter{Pruebas y Validación}\label{pruebas}
Una vez completada la implementación de la infraestructura de red hospitalaria simulada, resulta imprescindible realizar un proceso de pruebas y validación que permita 
verificar que la red funciona correctamente, cumple con los requisitos establecidos y garantiza los niveles de seguridad, segmentación y disponibilidad previstos en el diseño.
\\ \\
El propósito de este capítulo es detallar el conjunto de pruebas funcionales, de conectividad, de seguridad y de tolerancia a fallos realizadas sobre la red implementada en 
Cisco Packet Tracer. Estas pruebas permiten detectar posibles errores de configuración, comprobar la eficacia de las políticas de acceso y confirmar la correcta comunicación 
entre los diferentes dispositivos, subredes y hospitales.
\\ \\
Las pruebas se han clasificado según su naturaleza:
\begin{itemize}
    \item Pruebas de Conectividad.
    \item Pruebas de Seguridad.
    \item Pruebas de Disponibilidad.
    \item Pruebas de Protocolos de red.
\end{itemize}

Este capítulo documenta los resultados obtenidos en cada prueba, analizando su comportamiento frente a los objetivos planteados y proponiendo, en su caso, mejoras o ajustes sobre 
la configuración para optimizar la fiabilidad y seguridad de la red hospitalaria diseñada.

\section{Pruebas de Conectividad}
A continuación se muestran las pruebas de conectividad realizadas:
\begin{itemize}
    \item Conectividad entre dispositivos de áreas médicas con dispositivos de la UCI.
    \item Conectividad entre dispositivos de áreas médicas con dispositivos IoMT tipo 1.
    \item Bloqueo de tráfico entre dispositivos de áreas administrativas con dispositivos de áreas médicas.
    \item Bloqueo de tráfico entre dispositivos de áreas administrativas con dispositivos IoMT.
    \item Conectividad entre dispositivos de la UCI con dispositivos IoMT tipo 2.
    \item Conectividad entre el dispositivo autorizado de la UCI con dispositivos IoMT tipo 3.
    \item Bloqueo de tráfico entre dispositivos de la red de invitados con dispositivos de la red interna o IoMT.
    \item Conectividad entre dispositivos autorizados con servidores de archivos de otros hospitales.
    \item Conectividad entre dispositivos de la red interna con Internet.
    \item Conectividad entre dispositivos de la red de invitados con Internet.
\end{itemize}

\section{Pruebas de Seguridad}
A continuación se muestran las pruebas de seguridad realizadas:
\begin{itemize}
    \item Encriptación del tráfico entre hospitales.
    \item Correcto funcionamiento de DHCP Snooping.
    \item Correcto funcionamiento de la gestión remota mediante SSH.
    \item Configuración de autenticación en los dispositivos de red.
\end{itemize}

\section{Pruebas de Disponibilidad}
A continuación se muestran las pruebas de disponibilidad realizadas:
\begin{itemize}
    \item Tolerancia a fallos de HSRP en los routers.
    \item Tolerancia a fallos de HSRP en los switches L3.
    \item Tolerancia a fallos de EtherChannel.
    \item Tolerancia a fallos en la red de interconexión.
\end{itemize}

\section{Pruebas de Protocolos de Red}
A continuación se muestran las pruebas de protocolos de red realizadas:
\begin{itemize}
    \item Correcto funcionamiento de OSPF.
    \item Correcto funcionamiento de DHCP.
    \item Correcto funcionamiento de NAT.
\end{itemize}

\section{Resultados y Análisis}
