% !TEX root=MemoriaTFG.tex

\chapter{Metodología de Trabajo}\label{Metodología}


\section{Enfoque y Planificación del Proyecto}
Para garantizar el correcto desarrollo de este proyecto de diseño y simulación de una red hospitalaria, se optó por un enfoque metodológico secuencial y estructurado, 
basado en el modelo tradicional de desarrollo en cascada. Este modelo resulta especialmente adecuado para proyectos de carácter técnico y con una secuencia de tareas 
bien definida, como es el caso de la implementación de una infraestructura de red simulada, donde cada fase depende del correcto desarrollo de la anterior.\\

La metodología se articuló en torno a fases independientes y consecutivas, en las que se desarrollaron de forma separada y ordenada las distintas partes del proyecto: 
desde el análisis inicial de requisitos hasta las pruebas finales de validación, pasando por el diseño, la implementación y la configuración de los dispositivos y servicios 
de red.

\subsection{Enfoque de Trabajo Adoptado}
El trabajo se ha estructurado en cinco fases principales, organizadas secuencialmente:
\begin{enumerate}
    \item \textbf{Análisis de Requisitos:} recopilación y análisis de las necesidades funcionales, de conectividad y de seguridad que debía cubrir la red hospitalaria, 
    incluyendo las particularidades de la subred IoMT.
    \item \textbf{Diseño de la red:} elaboración de los diagramas de topología física y lógica, planificación del direccionamiento IP, definición de VLANs, políticas de 
    seguridad y segmentación.
    \item \textbf{Implementación de la infraestructura en Cisco Packet Tracer:} configuración de routers, switches, creación de VLANs, definición de ACLs y puesta en 
    funcionamiento de los servicios de red.
    \item \textbf{Pruebas y validación:} realización de pruebas de conectividad, comprobación de los servicios implementados y verificación de las políticas de seguridad aplicadas.
    \item \textbf{Documentación y cierre del proyecto:} redacción de las configuraciones, resultados de pruebas, y elaboración de la memoria técnica y académica del proyecto.
\end{enumerate}
Cada fase se abordó de forma secuencial, de modo que no se iniciaba una nueva hasta haber completado, revisado y validado la anterior, siguiendo así la filosofía del modelo en cascada.

\subsection{Planificación y Seguimiento}
Para asegurar el cumplimiento de la planificación establecida y el correcto desarrollo del proyecto, se realizaron reuniones de seguimiento quincenales con los tutores 
académicos. Estos encuentros resultaron clave para revisar los avances, corregir posibles errores detectados y planificar conjuntamente los siguientes pasos. Gracias a 
estas sesiones periódicas, se pudo ajustar la planificación en función de los resultados obtenidos en cada fase, resolviendo incidencias y mejorando progresivamente el 
diseño y configuración de la red. \\

A continuación se presenta una tabla con el cronograma del proyecto, que detalla las tareas realizadas y su duración estimada:

(tabla con cronograma del proyecto)

\section{Herramientas y Tecnologías Utilizadas}
Para el desarrollo y correcta gestión de este proyecto, se han empleado diversas herramientas tecnológicas que han permitido organizar las tareas, llevar a cabo las 
simulaciones de red y mantener un control estructurado sobre los cambios realizados en la configuración y documentación del proyecto. A continuación, se describen 
las herramientas utilizadas y su papel dentro del proyecto:

\subsection{Cisco Packet Tracer}
Para el diseño, simulación e implementación virtual de la red hospitalaria propuesta, se ha utilizado Cisco Packet Tracer, una herramienta de simulación de redes 
desarrollada por Cisco Systems que permite emular el comportamiento de dispositivos de red reales en entornos controlados.\\
Esta aplicación ha facilitado la creación de topologías de red personalizadas, la configuración de routers y switches, la asignación de direccionamientos IP, la 
implementación de VLANs y ACLs, así como la realización de pruebas de conectividad y seguridad. Además, Packet Tracer ha permitido visualizar de forma gráfica y 
detallada el tráfico de datos entre dispositivos, lo que ha sido fundamental para comprobar el correcto funcionamiento de la infraestructura antes de una hipotética 
implementación real.

\subsection{Git y GitHub}
Para llevar un control exhaustivo de las versiones de los archivos de configuración, documentación y esquemas de red, se ha empleado GitHub como sistema de control de 
versiones basado en la herramienta Git.
El uso de GitHub ha permitido mantener un histórico de los cambios realizados en el proyecto, facilitando así la recuperación de versiones anteriores en caso de necesidad 
y garantizando la trazabilidad de las modificaciones. Además, se ha utilizado como repositorio privado para almacenar las configuraciones de dispositivos, los diagramas de 
topología y los documentos de planificación, centralizando toda la información en un entorno accesible y seguro.

\subsection{Google Calendar}
Con el objetivo de organizar de forma eficiente el cronograma de trabajo, se ha empleado Google Calendar como herramienta de planificación y gestión temporal. Esta 
aplicación ha permitido establecer fechas límite, programar reuniones de seguimiento y distribuir las tareas en función de la carga de trabajo semanal. La posibilidad 
de añadir recordatorios y notificaciones ha resultado de gran utilidad para garantizar el cumplimiento de los hitos establecidos en el proyecto, manteniendo una correcta 
planificación y coordinación de las diferentes fases de desarrollo.

\subsection{SketchUp}
Para la elaboración del diagrama de topología física de la red hospitalaria, se ha empleado SketchUp, una herramienta de modelado en 3D que permite crear representaciones 
visuales detalladas de espacios y distribuciones físicas.
Gracias a esta aplicación, ha sido posible diseñar de forma visual la disposición de los distintos departamentos del hospital, algunos dispositivos de red, salas de servidores
y otros elementos relevantes, facilitando así la comprensión del diseño físico de la infraestructura. Este diagrama ha sido fundamental para complementar la documentación técnica y 
proporcionar una visión clara de la disposición de los equipos y la segmentación de la red en el entorno hospitalario.