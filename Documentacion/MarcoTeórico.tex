% !TEX root=ManualTFG.tex

\chapter{Marco Teórico}\label{MarcoTeórico}
Para comprender en profundidad el desarrollo de este proyecto y justificar las decisiones adoptadas durante su diseño e implementación, resulta imprescindible establecer 
una base teórica que aborde los conceptos, tecnologías y normativas implicadas. Este marco teórico tiene como finalidad proporcionar una visión general sobre las 
infraestructuras de red en entornos hospitalarios, su organización, los criterios de seguridad aplicables y las particularidades derivadas de la incorporación de tecnologías 
emergentes como el Internet of Medical Things (IoMT).\\

A lo largo de este capítulo se describirán los fundamentos de las redes LAN y VLAN, las características específicas de las redes hospitalarias, los requisitos de seguridad y 
segmentación, así como los retos que supone la gestión de dispositivos médicos conectados.

\section{Redes de Computadores en Entornos Hospitalarios}
El uso de redes de computadores en entornos hospitalarios constituye un pilar fundamental para el funcionamiento eficiente y seguro de los servicios sanitarios modernos. 
Los hospitales dependen en gran medida de infraestructuras de red que permitan la transmisión de información clínica, administrativa y de soporte, garantizando la 
disponibilidad, integridad y confidencialidad de los datos en todo momento. La correcta gestión de estas redes resulta esencial, ya que cualquier interrupción o fallo 
de seguridad puede afectar de forma directa a la calidad asistencial y, en casos críticos, incluso a la vida de los pacientes.\\

Dado el carácter crítico de estos entornos, una red hospitalaria debe ser diseñada y planificada cuidadosamente, considerando desde el inicio las necesidades presentes y 
futuras del centro, los riesgos potenciales y las medidas de seguridad necesarias. Una infraestructura bien segmentada y protegida no solo garantiza el correcto 
funcionamiento de los servicios asistenciales, sino que también contribuye a la confidencialidad de los datos clínicos y a la seguridad de los pacientes.\\

Además, la irrupción de tecnologías emergentes como el Internet of Medical Things (IoMT) ha supuesto un nuevo desafío para estas redes, aumentando la cantidad de dispositivos 
conectados y ampliando la superficie de exposición a posibles ciberamenazas, lo que ha obligado a reforzar las políticas de seguridad y segmentación en las infraestructuras 
hospitalarias modernas.

\section{Redes LAN y VLAN}
\subsection{Redes LAN en Entornos Hospitalarios}
Una Local Area Network (LAN) es una red de área local que permite la interconexión de dispositivos dentro de un ámbito geográfico limitado, como puede ser un edificio, 
planta hospitalaria o campus sanitario. Su finalidad es compartir recursos de red, dispositivos y servicios de manera rápida y segura, permitiendo la comunicación entre 
estaciones de trabajo, servidores, dispositivos médicos y sistemas de almacenamiento centralizados.\\

En entornos hospitalarios, las redes LAN constituyen la infraestructura básica sobre la que se sustentan tanto los sistemas administrativos como los servicios asistenciales. 
Gracias a la conectividad que proporcionan, se facilita el acceso a sistemas de información hospitalaria, historiales médicos electrónicos, servicios de radiología digital y 
monitorización médica en tiempo real. Además, la llegada de nuevas tecnologías como el Internet of Medical Things (IoMT) ha incrementado notablemente la densidad de 
dispositivos conectados a las redes LAN hospitalarias, exigiendo una mayor capacidad de gestión, segmentación y seguridad en estas infraestructuras.

\subsection{VLANs en Entornos Hospitalarios}
Para mejorar la eficiencia, la seguridad y la organización de las redes LAN, se recurre a la creación de Virtual LANs (VLANs). Una VLAN permite agrupar lógicamente 
dispositivos dentro de una misma red física, segmentando el tráfico de datos aunque se encuentren conectados al mismo switch o infraestructura física.\\

Esta segmentación se basa en criterios funcionales, organizativos o de seguridad, de manera que cada grupo de usuarios o dispositivos que comparte necesidades similares de 
comunicación se integra en una VLAN específica. Esto evita el tráfico innecesario entre áreas que no requieren comunicación directa, mejora la seguridad al aislar departamentos 
sensibles y optimiza el rendimiento de la red al reducir las colisiones y la congestión de tráfico.\\

En entornos hospitalarios, las VLAN se utilizan habitualmente para separar el tráfico de los distintos departamentos (Urgencias, Administración, Laboratorio, Consultas 
Externas, etc.) y también para aislar segmentos críticos como la red de dispositivos médicos o la subred específica de IoMT. Esta segmentación lógica permite, 
además, aplicar políticas de seguridad específicas mediante listas de control de acceso (ACLs), limitando la visibilidad y comunicación entre VLANs cuando sea necesario.\\

El uso de VLANs en entornos sanitarios aporta numerosos beneficios:
\begin{itemize}
    \item \textbf{Mejora de la seguridad:} al separar físicamente el tráfico sensible, como datos clínicos o monitorización de pacientes, del resto de la red.
    \item \textbf{Optimización del rendimiento:} al reducir la cantidad de tráfico broadcast y minimizar las colisiones de red.
    \item \textbf{Flexibilidad en la gestión:} al permitir reasignar dispositivos o usuarios a diferentes VLANs sin necesidad de modificar la infraestructura física.
    \item \textbf{Facilidad de escalado:} favoreciendo la incorporación de nuevos servicios o dispositivos, como los IoMT, sin comprometer la seguridad ni el rendimiento 
    de la red existente.
    \item \textbf{Aplicación de políticas de control de acceso más específicas y eficientes.}
\end{itemize}
Gracias a esta capacidad de segmentación y control, las VLAN se han convertido en un componente imprescindible para garantizar la seguridad, el buen funcionamiento y la 
disponibilidad de los servicios hospitalarios.

\section{Seguridad en Redes Hospitalarias}
La seguridad en redes hospitalarias se ha convertido en un aspecto esencial dentro de la gestión tecnológica sanitaria, dado que en estos entornos no solo se maneja información 
crítica de carácter personal y médico, sino que además se conectan dispositivos clínicos cuyo correcto funcionamiento puede incidir directamente en la seguridad y salud de 
los pacientes. La evolución hacia infraestructuras digitales más complejas y la incorporación masiva de dispositivos médicos conectados (IoMT) ha incrementado notablemente 
la superficie de exposición a posibles ciberataques, lo que obliga a diseñar políticas de seguridad específicas, adaptadas a las necesidades de este tipo de entornos.

\subsection{Principales Amenazas en Redes Hospitalarias}
Las redes hospitalarias, por su naturaleza, presentan una serie de vulnerabilidades que pueden ser explotadas si no se implementan las medidas adecuadas. Entre las amenazas 
más frecuentes se encuentran:
\begin{itemize}
    \item \textbf{Intercepción de datos sensibles:} accesos no autorizados a historiales médicos y datos clínicos personales.
    \item \textbf{Intrusiones externas e internas:} ataques desde el exterior o desde dentro de la propia red, aprovechando errores de configuración o dispositivos desprotegidos.
    \item \textbf{Ataques de denegación de servicio (DoS/DDoS):} que pueden dejar inoperativos servicios críticos.
    \item \textbf{Malware y ransomware:} capaces de bloquear el acceso a sistemas de información hospitalaria o alterar el funcionamiento de dispositivos clínicos.
    \item \textbf{Vulnerabilidades en dispositivos IoMT:} por sus recursos limitados y configuraciones menos robustas.
\end{itemize}

\section{Introducción a IoMT (Internet of Medical Things)}
Internet of Medical Things (IoMT) es una evolución natural de Internet of Things (IoT) aplicada al ámbito sanitario, que permite la interconexión de dispositivos médicos, 
sensores y sistemas de información clínica a través de redes seguras. Esta tecnología posibilita la monitorización remota de pacientes, el control en tiempo real de 
parámetros fisiológicos y la gestión eficiente de recursos hospitalarios, contribuyendo a mejorar la calidad asistencial y la toma de decisiones clínicas basadas en datos 
fiables y actualizados. \\

En la práctica hospitalaria, el IoMT se ha consolidado como una herramienta fundamental para optimizar los procesos sanitarios, incrementando la capacidad de respuesta ante situaciones
críticas y reduciendo la carga de trabajo del personal clínico. Gracias a la integración de sensores biomédicos, dispsoitivos portátiles y plataformas de gestión de datos, 
los profesionales sanitarios pueden disponer de información vital en tiempo real, lo que favorece diagnósticos más precisos y tratamientos personalizados. \\ 

Además, el IoMT desempeña un papel esencial en la mejora de la eficiencia operativa hospitalaria. Como se recoge en la literatura, su implementación permite localizar y gestionar
equipamiento médico, optimizar la trazabilidad de pacientes y activos, y mejorar la monitorización de entornos hospitalarios críticos, como quirófanos y unidades de cuidados intensivos.
Este ecosistema conectado se apoya en tecnologías de comunicación de baja potencia y alrgo alcance (LPWAN) como Sigfox, LoRa y NB-IoT, que proporcionan conectividad eficiente para
dispositivos médicos que requieren bajo consumo energético y cobertura extendida dentro y fuera de los centros sanitarios. \\

Desde el punto de vista arquitectónico, las soluciones IoMT han evolucionado hacia modelos distribuidos basados en edge/fog computing, donde los datos se procesan parcialemnte
en pasarelas inteligentes cercanas a los dispositivos, antes de enviarse a plataformas en la nube para su almacenamiento y análisis avanzado. Este enfoque permite reducir la 
latencia, mejorar la seguridad de los datos sensibles y aliviar la carga de tráfico hacia los servidores centrales, favoreciendo la continuidad asistencial en entornos hospitalarios
con elevada demanda de recursos. \\

El auge del IoMT también plantea desafíos en materia de seguridad, privacidad e interoperabilidad, dado que la cantidad de información médica gestionada por estos sistemas es altamente 
sensible y está sujeta a estrictos marcos normativos. \\

En definitiva, la implantación del IoMT en entronos hospitalrios representa una oportunidad estratégica para transformar la asistencia sanitaria, dotándola de mayor flexibilidad, 
capacidad predictiva y resiliencia frente a sistuaciones de crisis como la vivida durante la pandemia de COVID-19, donde estas tecnologías demostraron su potencial para mejorar la 
monitorización, la toma de decisiones y la gestión de recursos clínicos en tiempo real.
