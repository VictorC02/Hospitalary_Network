%!TeX root=MemoriaTFG.tex
\chapter{Anexos}\label{Anexo}

\section{Configuración Básica Dispositivos de Red}\label{anexo:confBas}
A continuación se muestra el fragmento de código bash utilizado para la configuración básica de los dispositivos de red:

\begin{lstlisting}[language=Bash, caption={Configuración Básica de los Dispositivos de Red}]
%Passwords
enable  
config t  
hostname  nombre_dispositivo 
enable password cisco  
no ip domain lookup  
banner motd #Solo Acceso Autorizado!!#  
line console 0  
password cisco  
login  
exit  
service password-encryption 

%SSH 
ip domain name cisco.net  
username admin password cisco  
crypto key generate rsa  
1024 
line vty 0 15  
login local  
transport input ssh
\end{lstlisting}

\section{Configuración VLANs en Switches de Acceso}\label{anexo:VLANSwAcc}
A continuación se muestra el fragmento de código bash utilizado para la configuración de las VLANs en los switches de acceso:

\begin{lstlisting}[language=Bash, caption={Configuración VLANs en Switches de Acceso}]
%Create VLAN
vlan xx 
name nombre_VLAN 
exit 

%Ports Configuration
interface range Fa0/1-2 
switchport mode trunk 
switchport trunk allowed vlan xx
exit 

interface range Fa0/3-24 
switchport mode access 
switchport access vlan xx
exit 
\end{lstlisting}

\section{Configuración VLANs en Switches de Distribución}\label{anexo:VLANSwDis}
A continuación se muestra el fragmento de código bash utilizado para la configuración de las VLANs en los switches de distribución:

\begin{lstlisting}[language=Bash, caption={Configuración VLANs en Switches de Distribución}]
%Create VLANs
vlan xx 
name nombre_VLAN1
exit 

vlan yy 
name nombre_VLAN2
exit 

vlan zz 
name nombre_VLAN_interconexion_Core
exit 

vlan nn 
name nombre_VLAN_interconexion_Distribuidores
exit 

%Ports Configuration
interface GigabitEthernet1/0/3 %Enlace con Switch Acceso 1
switchport mode trunk 
switchport trunk allowed vlan xx 
exit 

interface GigabitEthernet1/0/4 %Enlace con Switch Acceso 2
switchport mode trunk 
switchport trunk allowed vlan yy 
exit 

interface GigabitEthernet1/0/1 %Enlace con Switch Core 1
switchport mode trunk 
switchport trunk allowed vlan xx,yy,zz
exit 

interface GigabitEthernet1/0/6 %Enlace con Switch Core 2
switchport mode trunk 
switchport trunk allowed vlan xx,yy,zz
exit 

interface GigabitEthernet1/0/2 %Enlace con Switch Distribucion 1
switchport mode trunk 
switchport trunk allowed vlan xx,yy,nn
exit 

interface GigabitEthernet1/0/7 %Enlace con Switch Distribucion 2
switchport mode trunk 
switchport trunk allowed vlan xx,yy,nn
exit 
\end{lstlisting}

\section{Configuración VLANs en Switches Core}\label{anexo:VLANSwCo}
A continuación se muestra el fragmento de código bash utilizado para la configuración de las VLANs en los switches core:

\begin{lstlisting}[language=Bash, caption={Configuración VLANs en Switches Core}]
%Create VLANs
vlan xx 
name nombre_VLAN1
exit 

vlan yy 
name nombre_VLAN2
exit 

vlan nn
name nombre_VLAN3
exit 

vlan zz 
name nombre_VLAN_interconexion_Core
exit 

%Ports Configuration
interface GigabitEthernet1/0/3 %Enlace con Switch Distribucion 1
switchport mode trunk 
switchport trunk allowed vlan xx,zz 
exit 

interface GigabitEthernet1/0/4 %Enlace con Switch Distribucion 2
switchport mode trunk 
switchport trunk allowed vlan yy,zz
exit 

interface GigabitEthernet1/0/1 %Enlace con Switch Core
switchport mode trunk 
switchport trunk allowed vlan xx,yy,nn,zz
exit 
\end{lstlisting}

\section{Configuración EtherChannel}\label{anexo:etherchannel}
A continuación se muestra el fragmento de código bash utilizado para la configuración del EtherChannel.

\begin{lstlisting}[language=Bash, caption={Configuración EtherChannel}]
interface range fa0/1-2 %Enlaces al switch de acceso/distribucion
switchport mode trunk 
switchport trunk allowed vlan xx 
channel-group 1 mode active 
no shutdown 
exit

interface port-channel 1 
switchport mode trunk 
switchport trunk allowed vlan xx
no shutdown 
\end{lstlisting}

\section{Configuración Medidas Seguridad DMZ}\label{anexo:dmz}
A continuación se muestra el fragmento de código bash utilizado para la configuración de las medidas de seguridad básicas en las DMZ.

\begin{lstlisting}[language=Bash, caption={Configuración Medidas Seguridad DMZ}]
int range fa0/2-24 %Enlaces a dispositivos finales
switchport port-security maximum 1 
switchport port-security mac-address sticky 
switchport port-security violation shutdown 
\end{lstlisting}

\section{Configuración HSRP en Switches L3}\label{anexo:hsrpL3}
A continuación se muestra el fragmento de código bash utilizado para la configuración de la configuración de HSRP en los switches L3.

\begin{lstlisting}[language=Bash, caption={Configuración HSRP en Switches L3}]
interface vlan xx 
ip address 192.168.10.2 255.255.255.0 
standby xx ip 192.168.10.1 
standby xx priority 120 
standby xx preempt %Permite recuperar el control del servicio
no shutdown 
\end{lstlisting}

\section{Configuración HSRP en Routers}\label{anexo:hsrpRou}
A continuación se muestra el fragmento de código bash utilizado para la configuración de la configuración de HSRP en los routers.

\begin{lstlisting}[language=Bash, caption={Configuración HSRP en Routers}]
int range fa0/2-24 %Enlaces a dispositivos finales
switchport port-security maximum 1 
switchport port-security mac-address sticky 
switchport port-security violation shutdown 
\end{lstlisting}

\section{Configuración Direccionamiento IP Estático}\label{anexo:estatico}
A continuación se muestra el fragmento de código bash utilizado para el direccionamiento IP estático.

\begin{lstlisting}[language=Bash, caption={Configuración Direccionamiento IP Estático}]
int gig1/0/1
ip address 192.168.0.1 255.255.255.0
\end{lstlisting}

\section{Configuración DHCP Relay}\label{anexo:relay}
A continuación se muestra el fragmento de código bash utilizado para la configuración de DHCP Relay en switches L3.

\begin{lstlisting}[language=Bash, caption={Configuración Direccionamiento IP Estático}]
int vlan xx
ip helper-address 10.0.3.6 
\end{lstlisting}

\section{Configuración OSPF}\label{anexo:ospf}
A continuación se muestra el fragmento de código bash utilizado para la configuración de OSPF.

\begin{lstlisting}[language=Bash, caption={Configuración OSPF}]
ip routing 
router ospf 10  
#Poner todas las redes a las que esta conectado cada switch/router
network 10.2.1.0 0.0.0.7 area 0 
network 10.1.0.0 0.0.0.31 area 0
\end{lstlisting}

\section{Configuración RSTP}\label{anexo:rstp}
A continuación se muestra el fragmento de código bash utilizado para la configuración de RSTP.

\begin{lstlisting}[language=Bash, caption={Configuración RSTP}]
spanning-tree mode rapid-pvst 
\end{lstlisting}

\section{Configuración NAT}\label{anexo:nat}
A continuación se muestra el fragmento de código bash utilizado para la configuración de NAT.

\begin{lstlisting}[language=Bash, caption={Configuración NAT}]
%Outside Interfaces
interface serial1/0 
ip nat outside 

%Inside Interfaces
interface gig1/0.150
ip nat inside 

%Define ACLs
access-list 1 permit subred1 mascara_de_red1
access-list 1 permit subred2 mascara_de_red2
access-list 1 permit subred3 mascara_de_red3

access-list 2 permit subred1 mascara_de_red1
access-list 2 permit subred2 mascara_de_red2
access-list 2 permit subred3 mascara_de_red3

%Apply ACLs on Interfaces
ip nat inside source list 1 interface se0/0 overload 
ip nat inside source list 2 interface se1/0 overload 
\end{lstlisting}

\section{Configuración DHCP Snooping}\label{anexo:snooping}
A continuación se muestra el fragmento de código bash utilizado para la configuración de DHCP Snooping en los switches de acceso.

\begin{lstlisting}[language=Bash, caption={Configuración DHCP Snooping}]
%Active DHCP Snooping
ip dhcp snooping 
ip dhcp snooping vlan xx

%Packets Rate Limitation
int range fa0/1-24 
ip dhcp snooping limit rate 5 

%Trust Interfaces
int range fa0/1-2 
ip dhcp snooping trust 
\end{lstlisting}

\section{Configuración ACLs Switch IoMt Planta 3}\label{anexo:aclIoMTP3}
A continuación se muestra el fragmento de código bash utilizado para la configuración de ACLs en el switch IoMT de la planta 3.

\begin{lstlisting}[language=Bash, caption={Configuración ACLs Switch IoMt Planta 3}]
%Declare ACL
access-list 101 permit udp any any eq 67            
access-list 101 permit udp any any eq 68
access-list 101 permit ip 10.0.2.64 0.0.0.63 172.16.64.0 0.0.15.255
access-list 101 deny ip any any

access-list 102 permit udp any any eq 67            
access-list 102 permit udp any any eq 68
access-list 102 permit ip 172.16.64.0 0.0.15.255 10.0.2.64 0.0.0.63
access-list 102 deny ip any any

%Apply ACL on Interface
int vlan220
ip access-group 101 out
ip access-group 102 in

%Declare ACL
access-list 103 permit udp any any eq 67            
access-list 103 permit udp any any eq 68
access-list 103 permit ip host 10.0.2.70 172.16.128.0 0.0.15.255
access-list 103 deny ip any any
     
access-list 104 permit udp any any eq 67            
access-list 104 permit udp any any eq 68        
access-list 104 permit ip 172.16.128.0 0.0.15.255 host 10.0.2.70
access-list 104 deny ip any any

%Apply ACL on Interface
int vlan230
ip access-group 104 in
ip access-group 103 out
\end{lstlisting}

\section{Configuración ACLs Switch IoMt Plantas 0, 1 y 2}\label{anexo:aclIoMTP012}
A continuación se muestra el fragmento de código bash utilizado para la configuración de ACLs en el switch IoMT de las plantas 0, 1 y 2.

\begin{lstlisting}[language=Bash, caption={Configuración ACLs Switch IoMt Plantas 0, 1 y 2}]
%Declare ACL
access-list 120 permit udp any any eq 67            
access-list 120 permit udp any any eq 68
access-list 120 permit ospf any any
access-list 120 permit ip 10.0.2.64 0.0.0.63 172.16.48.0 0.0.15.255
access-list 120 permit ip 10.0.0.128 0.0.0.63 172.16.48.0 0.0.15.255
access-list 120 permit ip 10.0.0.192 0.0.0.63 172.16.48.0 0.0.15.255
access-list 120 permit ip 10.0.1.0 0.0.0.63 172.16.48.0 0.0.15.255
access-list 120 permit ip 10.0.1.64 0.0.0.63 172.16.48.0 0.0.15.255
access-list 120 permit ip 10.0.1.128 0.0.0.63 172.16.48.0 0.0.15.255
access-list 120 permit ip 10.0.1.192 0.0.0.63 172.16.48.0 0.0.15.255
access-list 120 deny ip any any


access-list 121 permit udp any any eq 67            
access-list 121 permit udp any any eq 68
access-list 121 permit ospf any any
access-list 121 permit ip 172.16.48.0 0.0.15.255 10.0.2.64 0.0.0.63 
access-list 121 permit ip 172.16.48.0 0.0.15.255 10.0.0.128 0.0.0.63 
access-list 121 permit ip 172.16.48.0 0.0.15.255 10.0.0.192 0.0.0.63 
access-list 121 permit ip 172.16.48.0 0.0.15.255 10.0.1.0 0.0.0.63 
access-list 121 permit ip 172.16.48.0 0.0.15.255 10.0.1.64 0.0.0.63 
access-list 121 permit ip 172.16.48.0 0.0.15.255 10.0.1.128 0.0.0.63 
access-list 121 permit ip 172.16.48.0 0.0.15.255 10.0.1.192 0.0.0.63
access-list 121 deny ip any any

%Apply ACL on Interface
int vlan 210
ip access-group 120 out
ip access-group 121 in
\end{lstlisting}

\section{Configuración ACLs Routers}\label{anexo:aclRout}
A continuación se muestra el fragmento de código bash utilizado para la configuración de ACLs en los routers.

\begin{lstlisting}[language=Bash, caption={Configuración ACLs Routers}]
%Declare ACL
access-list 110 permit udp any any eq 67            
access-list 110 permit udp any any eq 68
access-list 110 permit ospf any any
access-list 110 permit ip 192.168.0.0 0.0.127.255 195.136.17.0 0.0.0.3
access-list 110 permit ip 192.168.0.0 0.0.127.255 195.136.17.8 0.0.0.3
access-list 110 permit ip 192.168.0.0 0.0.127.255 195.136.17.16 0.0.0.3
access-list 110 permit ip 192.168.0.0 0.0.127.255 195.136.17.4 0.0.0.3
access-list 110 permit ip 192.168.0.0 0.0.127.255 195.136.17.12 0.0.0.3
access-list 110 permit ip 192.168.0.0 0.0.127.255 195.136.17.20 0.0.0.3
access-list 110 deny ip any any

access-list 111 permit udp any any eq 67            
access-list 111 permit udp any any eq 68
access-list 111 permit ospf any any
access-list 111 permit ip 195.136.17.0 0.0.0.3 192.168.0.0 0.0.127.255 
access-list 111 permit ip 195.136.17.8 0.0.0.3 192.168.0.0 0.0.127.255 
access-list 111 permit ip 195.136.17.16 0.0.0.3 192.168.0.0 0.0.127.255 
access-list 111 permit ip 195.136.17.4 0.0.0.3 192.168.0.0 0.0.127.255 
access-list 111 permit ip 195.136.17.12 0.0.0.3 192.168.0.0 0.0.127.255 
access-list 111 permit ip 195.136.17.20 0.0.0.3 192.168.0.0 0.0.127.255 
access-list 111 deny ip any any

%Apply ACL on Interface
int g4/0.190
ip access-group 110 in
ip access-group 111 out
\end{lstlisting}

\section{Configuración ACLs Switches Distribución}\label{anexo:acldistr}
A continuación se muestra el fragmento de código bash utilizado para la configuración de ACLs en los switches de distribución.

\begin{lstlisting}[language=Bash, caption={Configuración ACLs Switches Distribución}]
%Declare ACL
access-list 130 deny ip 10.0.0.0 0.0.0.63 10.0.0.128 0.0.0.63
access-list 130 deny ip 10.0.0.0 0.0.0.63 10.0.0.192 0.0.0.63
access-list 130 deny ip 10.0.0.0 0.0.0.63 10.0.1.0 0.0.0.63
access-list 130 deny ip 10.0.0.0 0.0.0.63 10.0.1.64 0.0.0.63
access-list 130 deny ip 10.0.0.0 0.0.0.63 10.0.1.128 0.0.0.63
access-list 130 deny ip 10.0.0.0 0.0.0.63 10.0.1.192 0.0.0.63
access-list 130 permit ip any any

%Apply ACL on Interface
int vlan10
ip access-group 130 in

%Declare ACL
access-list 131 deny ip 10.0.0.64 0.0.0.63 10.0.0.128 0.0.0.63
access-list 131 deny ip 10.0.0.64 0.0.0.63 10.0.0.192 0.0.0.63
access-list 131 deny ip 10.0.0.64 0.0.0.63 10.0.1.0 0.0.0.63
access-list 131 deny ip 10.0.0.64 0.0.0.63 10.0.1.64 0.0.0.63
access-list 131 deny ip 10.0.0.64 0.0.0.63 10.0.1.128 0.0.0.63
access-list 131 deny ip 10.0.0.64 0.0.0.63 10.0.1.192 0.0.0.63
access-list 131 permit ip any any

%Apply ACL on Interface
int vlan20
ip access-group 131 in

%Declare ACL
access-list 132 deny ip 10.0.2.0 0.0.0.63 10.0.0.128 0.0.0.63
access-list 132 deny ip 10.0.2.0 0.0.0.63 10.0.0.192 0.0.0.63
access-list 132 deny ip 10.0.2.0 0.0.0.63 10.0.1.0 0.0.0.63
access-list 132 deny ip 10.0.2.0 0.0.0.63 10.0.1.64 0.0.0.63
access-list 132 deny ip 10.0.2.0 0.0.0.63 10.0.1.128 0.0.0.63
access-list 132 deny ip 10.0.2.0 0.0.0.63 10.0.1.192 0.0.0.63
access-list 132 permit ip any any

%Apply ACL on Interface
int vlan90
ip access-group 132 in

%Declare ACL
access-list 133 deny ip 10.0.2.64 0.0.0.63 10.0.0.128 0.0.0.63
access-list 133 deny ip 10.0.2.64 0.0.0.63 10.0.0.192 0.0.0.63
access-list 133 deny ip 10.0.2.64 0.0.0.63 10.0.1.0 0.0.0.63
access-list 133 deny ip 10.0.2.64 0.0.0.63 10.0.1.64 0.0.0.63
access-list 133 deny ip 10.0.2.64 0.0.0.63 10.0.1.128 0.0.0.63
access-list 133 deny ip 10.0.2.64 0.0.0.63 10.0.1.192 0.0.0.63
access-list 133 permit ip any any

%Apply ACL on Interface
int vlan100
ip access-group 133 in

%Declare ACL
access-list 134 deny ip 10.0.2.128 0.0.0.63 10.0.0.128 0.0.0.63
access-list 134 deny ip 10.0.2.128 0.0.0.63 10.0.0.192 0.0.0.63
access-list 134 deny ip 10.0.2.128 0.0.0.63 10.0.1.0 0.0.0.63
access-list 134 deny ip 10.0.2.128 0.0.0.63 10.0.1.64 0.0.0.63
access-list 134 deny ip 10.0.2.128 0.0.0.63 10.0.1.128 0.0.0.63
access-list 134 deny ip 10.0.2.128 0.0.0.63 10.0.1.192 0.0.0.63
access-list 134 permit ip any any

%Apply ACL on Interface
int vlan110
ip access-group 134 in

%Declare ACL
access-list 135 deny ip 10.0.2.192 0.0.0.63 10.0.0.128 0.0.0.63
access-list 135 deny ip 10.0.2.192 0.0.0.63 10.0.0.192 0.0.0.63
access-list 135 deny ip 10.0.2.192 0.0.0.63 10.0.1.0 0.0.0.63
access-list 135 deny ip 10.0.2.192 0.0.0.63 10.0.1.64 0.0.0.63
access-list 135 deny ip 10.0.2.192 0.0.0.63 10.0.1.128 0.0.0.63
access-list 135 deny ip 10.0.2.192 0.0.0.63 10.0.1.192 0.0.0.63
access-list 135 permit ip any any

%Apply ACL on Interface
int vlan120
ip access-group 135 in
\end{lstlisting}

\section{Configuración ACLs Routers Interconexión}\label{anexo:aclint}
A continuación se muestra el fragmento de código bash utilizado para la configuración de ACLs en los routers de interconexión.

\begin{lstlisting}[language=Bash, caption={Configuración ACLs Routers Interconexión}]
%R-SonEspases-Man
access-list 110 permit ip 192.168.100.0 0.0.3.255 192.168.104.0 0.0.1.255
access-list 110 permit ip 192.168.100.0 0.0.3.255 192.168.106.0 0.0.0.127
access-list 110 permit ip 192.168.100.0 0.0.3.255 192.168.106.128 0.0.0.63

%SE-Inca
access-list 120 permit ip 192.168.100.0 0.0.3.255 192.168.108.0 0.0.1.255
access-list 120 permit ip 192.168.100.0 0.0.3.255 192.168.110.0 0.0.0.255
access-list 120 permit ip 192.168.100.0 0.0.3.255 192.168.111.0 0.0.0.63

%Inca-SE
access-list 120 permit ip 192.168.108.0 0.0.1.255 192.168.100.0 0.0.3.255 
access-list 120 permit ip 192.168.110.0 0.0.0.255 192.168.100.0 0.0.3.255 
access-list 120 permit ip 192.168.111.0 0.0.0.63 192.168.100.0 0.0.3.255 

%SE-SL
access-list 130 permit ip 192.168.100.0 0.0.3.255 192.168.112.0 0.0.1.255
access-list 130 permit ip 192.168.100.0 0.0.3.255 192.168.114.0 0.0.0.255
access-list 130 permit ip 192.168.100.0 0.0.3.255 192.168.115.0 0.0.0.127

%SL-SE
access-list 130 permit ip 192.168.112.0 0.0.1.255 192.168.100.0 0.0.3.255 
access-list 130 permit ip 192.168.114.0 0.0.0.255 192.168.100.0 0.0.3.255 
access-list 130 permit ip 192.168.115.0 0.0.0.127 192.168.100.0 0.0.3.255 

%Man-Inca
access-list 140 permit ip 192.168.104.0 0.0.1.255 192.168.108.0 0.0.1.255
access-list 140 permit ip 192.168.104.0 0.0.1.255 192.168.110.0 0.0.0.255
access-list 140 permit ip 192.168.104.0 0.0.1.255 192.168.111.0 0.0.0.63
access-list 140 permit ip 192.168.106.0 0.0.0.127 192.168.108.0 0.0.1.255
access-list 140 permit ip 192.168.106.0 0.0.0.127 192.168.110.0 0.0.0.255
access-list 140 permit ip 192.168.106.0 0.0.0.127 192.168.111.0 0.0.0.63
access-list 140 permit ip 192.168.106.128 0.0.0.63 192.168.108.0 0.0.1.255
access-list 140 permit ip 192.168.106.128 0.0.0.63 192.168.110.0 0.0.0.255
access-list 140 permit ip 192.168.106.128 0.0.0.63 192.168.111.0 0.0.0.63

%Inca-Man
access-list 140 permit ip 192.168.108.0 0.0.1.255 192.168.104.0 0.0.1.255 
access-list 140 permit ip 192.168.110.0 0.0.0.255 192.168.104.0 0.0.1.255 
access-list 140 permit ip 192.168.111.0 0.0.0.63 192.168.104.0 0.0.1.255 
access-list 140 permit ip 192.168.108.0 0.0.1.255 192.168.106.0 0.0.0.127 
access-list 140 permit ip 192.168.110.0 0.0.0.255 192.168.106.0 0.0.0.127 
access-list 140 permit ip 192.168.111.0 0.0.0.63 192.168.106.0 0.0.0.127 
access-list 140 permit ip 192.168.108.0 0.0.1.255 192.168.106.128 0.0.0.63 
access-list 140 permit ip 192.168.110.0 0.0.0.255 192.168.106.128 0.0.0.63 
access-list 140 permit ip 192.168.111.0 0.0.0.63 192.168.106.128 0.0.0.63 

%Inca-SL
access-list 150 permit ip 192.168.108.0 0.0.1.255 192.168.112.0 0.0.1.255 
access-list 150 permit ip 192.168.108.0 0.0.1.255 192.168.114.0 0.0.0.255 
access-list 150 permit ip 192.168.108.0 0.0.1.255 192.168.115.0 0.0.0.127 
access-list 150 permit ip 192.168.110.0 0.0.0.255 192.168.112.0 0.0.1.255 
access-list 150 permit ip 192.168.110.0 0.0.0.255 192.168.114.0 0.0.0.255 
access-list 150 permit ip 192.168.110.0 0.0.0.255 192.168.115.0 0.0.0.127 
access-list 150 permit ip 192.168.111.0 0.0.0.63 192.168.112.0 0.0.1.255 
access-list 150 permit ip 192.168.111.0 0.0.0.63 192.168.114.0 0.0.0.255 
access-list 150 permit ip 192.168.111.0 0.0.0.63 192.168.115.0 0.0.0.127

%SL-Inca
access-list 150 permit ip 192.168.112.0 0.0.1.255 192.168.108.0 0.0.1.255  
access-list 150 permit ip 192.168.114.0 0.0.0.255 192.168.108.0 0.0.1.255  
access-list 150 permit ip 192.168.115.0 0.0.0.127 192.168.108.0 0.0.1.255  
access-list 150 permit ip 192.168.112.0 0.0.1.255 192.168.110.0 0.0.0.255  
access-list 150 permit ip 192.168.114.0 0.0.0.255 192.168.110.0 0.0.0.255  
access-list 150 permit ip 192.168.115.0 0.0.0.127 192.168.110.0 0.0.0.255  
access-list 150 permit ip 192.168.112.0 0.0.1.255 192.168.111.0 0.0.0.63  
access-list 150 permit ip 192.168.114.0 0.0.0.255 192.168.111.0 0.0.0.63  
access-list 150 permit ip 192.168.115.0 0.0.0.127 192.168.111.0 0.0.0.63 

%Man-SL
access-list 160 permit ip 192.168.104.0 0.0.1.255 192.168.112.0 0.0.1.255
access-list 160 permit ip 192.168.104.0 0.0.1.255 192.168.114.0 0.0.0.255
access-list 160 permit ip 192.168.104.0 0.0.1.255 192.168.115.0 0.0.0.127
access-list 160 permit ip 192.168.106.0 0.0.0.127 192.168.112.0 0.0.1.255
access-list 160 permit ip 192.168.106.0 0.0.0.127 192.168.114.0 0.0.0.255
access-list 160 permit ip 192.168.106.0 0.0.0.127 192.168.115.0 0.0.0.127
access-list 160 permit ip 192.168.106.128 0.0.0.63 192.168.112.0 0.0.1.255
access-list 160 permit ip 192.168.106.128 0.0.0.63 192.168.114.0 0.0.0.255
access-list 160 permit ip 192.168.106.128 0.0.0.63 192.168.115.0 0.0.0.127

%SL-Man
access-list 160 permit ip 192.168.112.0 0.0.1.255 192.168.104.0 0.0.1.255 
access-list 160 permit ip 192.168.114.0 0.0.0.255 192.168.104.0 0.0.1.255 
access-list 160 permit ip 192.168.115.0 0.0.0.127 192.168.104.0 0.0.1.255 
access-list 160 permit ip 192.168.112.0 0.0.1.255 192.168.106.0 0.0.0.127 
access-list 160 permit ip 192.168.114.0 0.0.0.255 192.168.106.0 0.0.0.127 
access-list 160 permit ip 192.168.115.0 0.0.0.127 192.168.106.0 0.0.0.127 
access-list 160 permit ip 192.168.112.0 0.0.1.255 192.168.106.128 0.0.0.63 
access-list 160 permit ip 192.168.114.0 0.0.0.255 192.168.106.128 0.0.0.63 
access-list 160 permit ip 192.168.115.0 0.0.0.127 192.168.106.128 0.0.0.63 
\end{lstlisting}

\section{Configuración VPN entre Son Espases y Manacor}\label{anexo:vpn}
A continuación se muestra el fragmento de código bash utilizado para la configuración de la VPN entre Son Espases y Manacor.

\begin{lstlisting}[language=Bash, caption={Configuración VPN entre Son Espases y Manacor}]
%R-Son_Espases
crypto isakmp policy 60
encryption aes 256
authentication pre-share
group 5
exit
crypto isakmp key vpnpa55 address 192.168.103.178
crypto ipsec transform-set VPN-SET6 esp-aes esp-sha-hmac
crypto map VPN-MAP6 60 ipsec-isakmp
description VPN connection to Son Llatzer.
set peer 192.168.103.178
set transform-set VPN-SET6
match address 160

%Apply VPN Config on Interface
interface S0/3/1
crypto map VPN-MAP6

%R-Manacor
crypto isakmp policy 60
encryption aes 256
authentication pre-share
group 5
exit
crypto isakmp key vpnpa55 address 192.168.103.177
crypto ipsec transform-set VPN-SET6 esp-aes esp-sha-hmac
crypto map VPN-MAP6 60 ipsec-isakmp
description VPN connection to Manacor.
set peer 192.168.103.177
set transform-set VPN-SET6
match address 160
exit

%Apply VPN Config on Interface
interface S0/3/0
crypto map VPN-MAP6
\end{lstlisting}
