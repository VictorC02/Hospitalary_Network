%!TeX root=MemoriaTFG.tex

\chapter{Resum}
En l'actualitat, el correcte disseny i l'adequada seguretat de les xarxes hospitalàries són aspectes crítics per garantir la continuïtat assistencial,
la privacitat de les dades clíniques i la disponibilitat dels sistemes mèdics. Atès que els hospitals gestionen informació sensible i dispositius
vitals que requereixen una connectivitat estable i protegida, resulta imprescindible implementar infraestructures de xarxa segures, segmentades i adaptades a les
particularitats d'aquest entorn. A més, la creixent incorporació de dispositius mèdics (IoMT) augmenta la superfície d'exposició i demanda
estratègies més restrictives per a la seva protecció. \\

En aquest projecte es presenta el disseny i la simulació de dues xarxes hospitalàries utilitzant Cisco Packet Tracer: una proposa una xarxa detallada per a l'Hospital Son Espases i l'altra
proposa una xarxa d'interconnexió entre els quatre hospitals més grans de Mallorca (Son Espases, Hospital de Manacor, Hospital Comarcal d'Inca i Son Llàtzer), incorporant diverses mesures de seguretat tant a
nivell físic com lògic. La solució proposada segmenta la xarxa mitjançant VLANs per a aïllar els diversos departaments de l'hospital i estableix polítiques de
seguretat mitjançant llistes de control d'accés (ACLs), a més d'incloure redundància tant a nivell de maquinari com a
nivell d'enllaç, així com protocols i serveis de xarxa essencials com DHCP o NAT. Com a element diferencial, s'ha implementat una subxarxa específica per als dispositius IoMT, configurada amb restriccions i mesures de seguretat
avançades que limiten la seva interacció amb la resta de la infraestructura, minimitzant així els riscos associats a la seva connectivitat.