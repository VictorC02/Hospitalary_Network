%!TeX root=MemoriaTFG.tex

\chapter{Análisis }\label{análisis}

\section{Interesados}
Los interesados son aquellas personas o entidades que tienen un interés en el proyecto y pueden influir en su desarrollo o verse afectadas por él. En este caso, 
los interesados son los siguientes:

\begin{itemize}
    \item \textbf{Personal médico y sanitario:} Son los usuarios principales de los sistemas clínicos conectados a la red. Utilizan aplicaciones para la gestión de 
    historiales médicos, diagnósticos, prescripciones y monitorización en tiempo real de los pacientes. Para este colectivo, la red debe garantizar alta disponibilidad, 
    bajo retardo y confidencialidad de los datos clínicos, ya que cualquier interrupción puede afectar directamente a la atención sanitaria.
    \item \textbf{Personal administrativo:} Encargados de la gestión de citas, facturación, expedientes, inventario y coordinación interna del hospital. Aunque sus tareas 
    no están directamente relacionadas con la atención clínica, requieren acceso constante a sistemas de información conectados a la red. Su trabajo depende de la fiabilidad 
    de los servicios internos como bases de datos y aplicaciones de gestión.
    \item \textbf{Departamento de IT:} Responsables del mantenimiento, configuración y supervisión de la infraestructura de red. Este grupo necesita una red segura, escalable 
    y fácilmente monitorizable, así como herramientas para la detección de fallos, gestión de dispositivos IoMT y control de accesos.
    \item \textbf{Dirección del hospital:} Interesada en que la red contribuya a mejorar la eficiencia operativa del centro, optimice los recursos y garantice el cumplimiento 
    de la legislación vigente, especialmente en lo relativo a la protección de datos (\ac{LOPDGDD} y \ac{RGPD}). También se preocupan por el coste y la sostenibilidad del sistema a 
    largo plazo.
    \item \textbf{Pacientes:} Interesados en que la red de invitados funcione correctamente y no tenga fallos de seguridad. Su experiencia asistencial mejora cuando los 
    procesos internos son ágiles, seguros y eficientes. La red hospitalaria debe garantizar que su información médica esté protegida, que los dispositivos de monitorización 
    funcionen en tiempo real y que la atención sea fluida y sin errores derivados de caídas de red.
\end{itemize}

\section{Requisitos del Sistema}
Para garantizar el correcto diseño, funcionamiento y seguridad de la infraestructura de red hospitalaria simulada en este proyecto, se han definido una serie de requisitos 
que determinan las condiciones que debe cumplir el sistema. Estos requisitos se clasifican en funcionales, no funcionales, de conectividad y de seguridad, abarcando tanto 
los aspectos técnicos como los operativos de la red.

\subsection{Requisitos Funcionales}
Son aquellos requisitos que definen las funciones y servicios que debe ofrecer la infraestructura de red para satisfacer las necesidades del entorno hospitalario y los 
dispositivos conectados.
\begin{itemize}
    \item Cada hospital debe disponer de tres subredes diferenciadas: una para invitados, una para dispositivos IoMT y otra para la red interna hospitalaria.
    \item Cada hospital debe estar dividido en VLANs por departamentos, asegurando la separación lógica de las áreas médicas, administrativas, de investigación y de enfermería.
    \item La red debe permitir la asignación dinámica de direcciones IP mediante servidores DHCP en cada subred.
    \item Debe garantizar la resolución de nombres internos mediante DNS.
    \item Los dispositivos autorizados (\ac{PC}s del departamento de IT) deben poder acceder a servidores de datos y dispositivos médicos críticos entre hospitales.
    \item Los dispositivos médicos IoMT deben disponer de una subred propia con dos servidores DHCP, uno de ellos de respaldo.
    \item Deben implementarse servicios de NAT para acceso a Internet desde las subredes del hospital.
    \item Se debe permitir gestión remota segura de los dispositivos de red mediante SSH.
    \item La red debe contar con redundancia de enlaces y gateways mediante EtherChannel y HSRP.
    \item Tanto la red intrahospitalaria como la interhospitalaria debe estar configurada de tal forma que haya configuraciones dinámicas de enrutamiento utilizando OSPF.
    \item Las comunicaciones desde el exterior de la red hacia el interior tienen que pasar por una interfaz diferente de las que van desde el interior hacia el exterior.
\end{itemize}
\subsection{Requisitos No Funcionales}
Son aquellos que definen condiciones de calidad, operativas o de gestión que debe cumplir el sistema, sin especificar funciones concretas.
\begin{itemize}
    \item La infraestructura debe garantizar una alta disponibilidad, con redundancia en enlaces y puntos críticos.
    \item La red debe estar diseñada de forma modular y escalable, permitiendo incorporar nuevos departamentos o dispositivos sin afectar al rendimiento.
\end{itemize}
\subsection{Requisitos de Disponibilidad y Redundancia}
Establecen las condiciones que debe cumplir la red para garantizar su funcionamiento continuo y la disponibilidad de los servicios críticos. En términos generales se establecen los siguientes requisitos:
\begin{itemize}
    \item Los enlaces entre switches L2 y L3 deben contar con redundancia física para evitar puntos únicos de fallo.
    \item Los routers principales deben estar configurados en alta disponibilidad para tolerar fallos de hardware o enlaces.
    \item Los servidores DHCP, de la red IoMT, deben contar con un servidor de respaldo para garantizar la asignación continua de direcciones IP.
    \item La infraestructura debe permitir la monitorización continua del estado de los dispositivos y enlaces para detectar fallos proactivamente.
\end{itemize}

\subsection{Requisitos de Conectividad}
Definen las condiciones relacionadas con la transmisión de datos y la comunicación entre dispositivos y servicios dentro de la red, en términos generales.
\begin{itemize}
    \item Todos los dispositivos de la red deben tener conectividad con cualquier dispositivo de la red.
\end{itemize}

\subsection{Requisitos de Seguridad}
\label{subsec: TiposIoMT}
Establecen las condiciones que debe cumplir la red para proteger su infraestructura, los datos transmitidos y los servicios prestados. En términos generales se establecen los siguientes requisitos:
\begin{itemize}
    \item Se debe implementar una segmentación lógica mediante VLANs para aislar departamentos, subredes y servicios críticos.
    \item Deben configurarse listas de control de acceso (ACLs) para:
    \begin{itemize}
        \item Bloquear el tráfico desde la red de invitados hacia la red interna y la red de dispositivos IoMT.
        \item Restringir el acceso a la base de datos de cada hospital, permitiendo únicamente a un PC autorizado del departamento de IT de cada hospital acceder a ella.
        \item Restringir el acceso a los dispositivos IoMT según su clasificación y función, limitando la conectividad entre ellos.
        \item Permitir el acceso a los dispositivos de la red interna únicamente a los dispositivos autorizados de las áreas de servicios quirúrgicos, médicos, centrales y \ac{UCI}.
        \item Solo permitir el tráfico desde Internet hasta el servidor web.
    \end{itemize}
    \item La gestión de dispositivos de red debe realizarse mediante conexiones seguras (SSH).
    \item Todos los dispositivos de red deben contar con una configuración de autenticación, con contraseñas seguras, para evitar accesos no autorizados.
    \item La interconexión entre hospitales debe realizarse a través de enlaces seguros, utilizando VPNs o túneles cifrados para proteger la información transmitida.
\end{itemize}
Para establecer los requisitos de seguridad específicos para los dispositivos IoMT, se deben considerar las siguientes clasificaciones de dispositivos:
\begin{itemize}
    \item \textbf{Dispositivos IoMT Comunes (Tipo 1):} Son aquellos dispositivos comunes para la atención médica que pueden ser utilizados por el personal 
    médico. Estos dispositivos pueden incluir sistemas de monitorización remota de pacientes, sensores de localización, dispositivos implantables, etc.
    \item \textbf{Dispositivos IoMT Importantes UCI (Tipo 2):} Son aquellos dispositivos que no son críticos dentro de la UCI, es decir, que su función principal no es vital de forma 
    inmediata. Estos dispositivos incluyen bombas de vacío para heridas, lámparas de fototerapia, dispositivos de rehabilitación, etc \cite{IoMT-UCI}.
    \item \textbf{Dispositivos IoMT Críticos UCI (Tipo 3):} Son aquellos dispositivos que son críticos dentro de la UCI, es decir, son aquellos que permiten la monitorización y soporte
    vital de pacientes en estado grave, donde la vigilancia constante y la intervención inmediata pueden marcar la diferencia entre la vida y la muerte, y que por tanto necesitan medidas
    de seguridad extras. Estos dispositivos incluyen monitores cardíacos y de signos vitales, ventiladores mecánicos, bombas de infusión, sistemas de hemodiálisis, etc\cite{IoMT-UCI}.
\end{itemize}
A continuación se detallan los requisitos de seguridad específicos para los dispositivos IoMT:
\begin{itemize}
    \item Los dispositivos IoMT deben estar aislados en una subred propia para evitar interferencias con la red interna y de invitados.
    \item Los dispositivos IoMT deben contar con un servidor propio de DHCP para la asignación de direcciones IP y un servidor de respaldo para garantizar la continuidad del servicio.
    \item Los dispositivos IoMT Tipo 1 solo deben tener conectividad con los dispositivos de las áreas médicas de servicios quirúrgicos, médicos y centrales, 
    además de los dispositivos del departamento de la UCI.
    \item Los dispositivos IoMT Tipo 2 deben tener conectividad únicamente con los dispositivos del departamento de la UCI.
    \item Los dispositivos IoMT Tipo 3 solo pueden tener conectividad con un único dispositivo autorizado del departamento de la UCI.
\end{itemize}
