%!TeX root=MemoriaTFG.tex

\chapter{Acrónimos} %Respectau títol del capítol.
%
% Per utilitzar els acrònims es recomana fer un poc 
% de recerca bibliogràfica per entendre com 
% funcionen. Concretament podeu llegir el manual
% que teniu dins el vostre sistema.
% La comanda `texdoc acronym` hauria de mostrar-lo.
%
\begin{acronym}

    \acro{ACL}{Access Control List}

    \acro{IoMT}{Internet of Medical Things}

    \acro{VLAN}{Virtual Local Area Network}

    \acro{IP}{Internet Protocol}

    \acro{DHCP}{Dynamic Host Configuration Protocol}

    \acro{DNS}{Domain Name System}

    \acro{NAT}{Network Address Translation}

    \acro{HSRP}{Hot Standby Router Protocol}

    \acro{SSH}{Secure Shell}

    \acro{ICMP}{Internet Control Message Protocol}

    \acro{MAC}{Media Access Control}

    \acro{IEEE}{Institute of Electrical and Electronics Engineers}

    \acro{LPWAN}{Low Power Wide Area Network}

    \acro{LOPDGDD}{Ley Orgánica de Protección de Datos y Garantía de los Derechos Digitales}

    \acro{RGPD}{Reglamento General de Protección de Datos}

    \acro{VPN}{Virtual Private Network}

    \acro{OSPF}{Open Shortest Path First}

    \acro{UCI}{Unidad de Cuidados Intensivos}

    \acro{IoT}{Internet of Things}

    \acro{DMZ}{Demilitarized Zone}

    \acro{LSDB}{Link-state Database}

    \acro{LSA}{Link-state Advertisements}

    \acro{SPF}{Shortest Path First}

    \acro{PAgP}{Port Aggregation Protocol}

    \acro{LACP}{Link Aggregation Control Protocol}

    \acro{PAT}{Port Address Translation}

    \acro{MITM}{Man In The Middle}

    \acro{IPSec}{Internet Protocol Security}

    \acro{LPWAN}{Low Power Wide Area Network}

    \acro{NB-IoT}{NarrowBand Internet of Things}

    \acro{LoRa}{Long Range}

    \acro{PC}{Personal Computer}

    \acro{RFC}{Request for Comments}

    \acro{ISP}{Internet Service Provider}

    \acro{AP}{Access Point}
    
\end{acronym}
