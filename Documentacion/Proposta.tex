% !TEX root=ManualTFG.tex

\chapter{La proposta de treball de final de grau}\label{proposta}

\section{Principis bàsics}
En un treball de final de grau ben planificat, la proposta de \ac{TFG} constituirà el primer document formal relatiu al projecte. A més, especialment amb la introducció del contracte docent relatiu al \ac{TFG}, la proposta assoleix una importància cabdal ja que aquest és el document sobre el qual estudiant, director i \ac{EPS} adquireixen compromisos respecte de la duració màxima del treball, tasques de supervisió, disponibilitat de materials, \ldots

La redacció d'una proposta ben definida i consensuada entre director i alumne suposa avantatges importants com, per exemple:
\begin{itemize}
\item Garanties de que, abans d'iniciar el desenvolupament del \ac{TFG}, l'alumne comprèn perfectament el problema que es vol abordar i és capaç de contextualitzar-lo, entenent com la temàtica del seu \ac{TFG} lliga amb la resta de coneixements de l'àrea.

\item Disponibilitat d'una descripció precisa dels objectius del \ac{TFG} i d'una estimació dels resultats que se n'esperen.

\item Visualització del full de ruta del \ac{TFG} que ha de permetre, tant al director com a l'alumne, l'avaluació del progrés en l'execució del projecte i en l'assoliment dels objectius.
\end{itemize}

En la major part dels casos, la proposta formal d'un \ac{TFG} es redactarà a partir d'una proposta de tema provinent d'un professor o d'un grup de recerca. Aquesta proposta de tema descriurà breument el context del treball, el problema adreçat i els objectius que s'intenten assolir. A més, hauria de proporcionar les referències i eines bàsiques a utilitzar i explicitar, també, els coneixements previs i habilitats recomanades per poder dur a terme el \ac{TFG}. El més desitjable seria que totes les propostes de temes de \ac{TFG} es fessin públiques a través de la web de l'\ac{EPS}, tanmateix és probable que també n'apareguin als taulers d'anuncis, portes dels laboratoris de recerca o les portes dels despatxos dels professors. És responsabilitat dels alumnes estar atents a la publicació d'aquests anuncis.

Es pot donar el cas d'alumnes que prefereixin realitzar el \ac{TFG} en una empresa o alumnes que estiguin especialment interessats en un àrea de coneixement concreta i ells mateixos vulguin proposar un tema de \ac{TFG}. En qualsevol d'aquests casos és important que trobin un supervisor i que discuteixin amb ell la preparació de la proposta formal de \ac{TFG}. En el cas de \ac{TFG}s desenvolupats en empreses serà important també comptar amb el vist-i-plau del director/tutor dintre de l'empresa.

El més habitual serà que l'alumne interessat en una proposta de tema de \ac{TFG} comenci llegint les referències bàsiques subministrades per tal d'assolir uns coneixements bàsics sobre el context general del tema proposat i poder plantejar el problema que es vol abordar i els objectius concrets que es persegueixen. Això, comptant amb el suport del supervisor del \ac{TFG}, ha de permetre passar a l'etapa de redacció de la proposta formal del \ac{TFG}. L'elaboració de la proposta pot implicar l'estudi de conceptes que no s'han cobert en els estudis de grau, la cerca de referències bibliogràfiques addicionals a les subministrades a la proposta del professor o l'avaluació d'eines \emph{hardware} i/o \emph{software} que no s'havien utilitzat abans.

La transició des de la proposta de tema del professor cap a la proposta formal de \ac{TFG} per part de l'alumne pot servir, també, per adaptar l'enunciat general d'un problema a un enunciat més específic que reculli les inquietuds de l'estudiant. A més, la proposta d'un professor o d'un grup de recerca pot acabar derivant en més d'una proposta de \ac{TFG} si hi ha més d'un alumne interessat en el tema i és possible estructurar-la en diversos \ac{TFG}s.

Finalment, és important recordar que la proposta de \ac{TFG} s'adjuntarà a un contracte docent entre director, alumne i \ac{EPS} i, per tant, ha d'estar consensuada entre les tres parts.


\section{La proposta de tema}

El format de la proposta de tema del professor serà bastant variable ja que dependrà molt del tipus de projecte concret a desenvolupar i també de l'\emph{estil} de cada professor. En qualsevol cas hauria de contenir informació suficient per a que l'alumne pugui tenir una idea prou clara de en què consisteix el projecte i quins objectius es persegueixen, de quins coneixements li faran falta per afrontar-ho i de quines són les referències bàsiques del tema, que poden servir com a punt de partida per entendre altres referències necessàries per redactar la proposta de \ac{TFG}.

Així, doncs, alguns elements importants que haurien de quedar reflectits a la proposta de tema són:
\begin{itemize}
\item Contextualització: La proposta de tema hauria d'incloure una breu introducció al context del \ac{TFG}, però suficientment detallada per facilitar què l'alumne pugui relacionar la feina a fer amb els coneixements adquirits al llarg dels estudis grau.
\item Definició d'objectius: S'haurien d'explicitar els objectius que es perseguiran en el projecte. El grau de concreció dels objectius dependrà del projecte en si: a vegades la proposta de tema ja suggerirà els resultats concrets que es pretén obtenir i en altres ocasions serà feina de l'alumne definir els objectius concrets del seu \ac{TFG}.
\item Bibliografia: A l'hora de proporcionar una llista de referències bibliogràfiques bàsiques s'ha de tenir en compta el nivell de coneixements dels alumnes potencialment interessats
    en el tema de \ac{TFG}. És molt probable que aquesta bibliografia bàsica estigui principalment composta de capítols de llibres i articles introductoris publicats en revistes i congressos.
\item Pre-requisits: La proposta de tema de \ac{TFG} també hauria d'especificar si el \ac{TFG} té pre-requisits en forma d'assignatures que és necessari haver cursat (o estar cursant), habilitats (e.g. programació de baix nivell, càlcul matemàtic) o interessos particulars dels alumnes.
\end{itemize}

La transició des de proposta de tema publicada per un professor o per un grup de recerca cap a la proposta de \ac{TFG} que ha de redactar l'estudiant, amb el suport del seu supervisor, és previsible que impliqui les passes següents:
\begin{itemize}
\item Immersió en el tema: L'alumne haurà de ser capaç de relacionar el tema del \ac{TFG} amb les assignatures que ha cursat durant els seus estudis de grau. Aquest procés ha d'ajudar a l'alumne a prendre consciència de quins coneixements bàsics li faran falta per dur a terme el \ac{TFG}. La bibliografia bàsica proporcionada a la proposta de tema ha de servir per a que l'alumne adquireixi els coneixements fonamentals per contextualitzar el seu treball.
\item Definició d'objectiu: Un cop assimilats els continguts de les referències introductòries s'ha de ser capaç de definir de manera ben concreta els objectius del \ac{TFG}. Per això pot ser necessària la consulta d'altres referències més especialitzades i/o la discussió d'alguns aspectes del treball amb el supervisor. Aquesta definició d'objectius també ha de servir, si és el cas, per incorporar a la proposta els interessos personals del projectista.
\item Programació temporal: Tenir objectius ben delimitats i ser conscients dels coneixements necessaris per dur-los a terme ha de permetre a l'alumne fer una programació temporal realista del \ac{TFG}. Serà important comprovar que aquesta programació temporal s'ajusta a la càrrega de feina que s'havia previst a la proposta de tema i a la càrrega en crèdits ECTS del \ac{TFG}. Si aquest no és el cas, serà necessari tornar a avaluar els objectius del projecte per determinar si aquests s'han d'ampliar o reduir. A l'hora de fer aquesta programació temporal és fonamental que el projectista tingui en compte la càrrega de feina addicional al \ac{TFG} (assignatures, pràctiques en empresa, \ldots). Els diagrames de PERT o GANT són eines útils per realitzar aquesta programació temporal.
\end{itemize}
Com ja s'ha mencionat abans, la proposta de tema pot donar lloc a més d'una proposta de \ac{TFG} si, per exemple, hi ha més d'un alumne interessat en el tema proposat. En aquest cas serà
responsabilitat del supervisor verificar que les diferents propostes, encara que parteixin d'un \emph{background} comú, tenen objectius concrets ben diferenciats.

\section{Estructura de la proposta de Treball Final de Grau}
En general l'estructura de la proposta de \ac{TFG} inclourà:
\begin{itemize}
\item \emph{Títol}: El títol ha de descriure de la manera més precisa possible el treball a realitzar. Aquest títol no ha de coincidir necessàriament amb el títol definitiu que apareixerà a la memòria.
\item \emph{Paraules clau}: La llista de paraules clau ha d'incloure els conceptes fonamentals que formen la base del \ac{TFG}. En cas de que eventualment s'arribi a disposar d'una base de dades de \ac{TFG}s, aquesta llista serà molt útil a l'hora de fer-hi cerques en funció dels temes tractats.
\item \emph{Context}: Aquesta secció ha de servir per situar el \ac{TFG} dins una determinada àrea de coneixement, per fer una descripció concisa sobre l'estat de l'art del tema sobre el que tractarà el \ac{TFG} i per aclarir els motius que han portat a la realització d'aquest treball. Hauria de servir, també, per introduir el plantejament general del problema que es pretén abordar. Els conceptes clau del projecte han d'aparèixer en aquesta secció definits de manera clara per tal de resoldre possibles ambigüitats i malentesos entre alumne i supervisor. És previsible que aquesta secció, juntament amb la dedicada als objectius, constitueixin la base del primer capítol de la memòria del \ac{TFG}.
\item \emph{Objectius}: En els objectius s'han de concretar les fites del treball tot indicant les qüestions específiques adreçades en el \ac{TFG} i els mètodes que s'empraran per donar-hi resposta. És important que siguin objectius mesurables, és a dir, que el seu assoliment sigui constatable.
\item \emph{Programació temporal}: La programació temporal descriurà el ritme i ordre en que s'han d'anar realitzant les diferents activitats. El nivell de detall de la planificació dependrà del que acordin alumne i supervisor quant a la periodicitat de les reunions des supervisió.
\item \emph{Eines}: Aquí s'explicitaran aquelles eines \emph{hardware} i/o \emph{software} que es faran servir per dur a terme el \ac{TFG}.
\item \emph{Bibliografia}. La bibliografia inclourà totes les referències bibliogràfiques rellevants en la redacció de la proposta de \ac{TFG}, tant les proporcionades en la proposta de tema com aquelles que s'hagin pogut descobrir durant els processos de documentació i redacció dels antecedents i objectius de la proposta final.
\end{itemize}
