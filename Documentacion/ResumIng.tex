%!TeX root=MemoriaTFG.tex

\chapter{Summary}
Currently, the proper design and adequate security of hospital networks are critical aspects to ensure continuity of care, the privacy of clinical data, and the availability 
of medical systems. Since hospitals handle sensitive information and vital devices that require stable and protected connectivity, it is essential to implement secure, 
segmented network infrastructures tailored to the specific characteristics of this environment. Furthermore, the increasing incorporation of medical devices (IoMT) expands 
the exposure surface and demands more restrictive strategies for their protection. \\

This project presents the design and simulation of two hospital networks using Cisco Packet Tracer: one proposes a detailed network for Hospital Son Espases, and the other 
proposes an interconnection network between the four largest hospitals in Mallorca (Son Espases, Hospital de Manacor, Hospital Comarcal de Inca, and Son Llàtzer), incorporating 
various security measures at both the physical and logical levels. The proposed solution segments the network using VLANs to isolate the different hospital departments and 
establishes security policies through access control lists (ACLs), in addition to including redundancy at both the hardware and link levels, as well as essential network 
protocols and services such as DHCP and NAT. As a distinguishing feature, a specific subnet for IoMT devices has been implemented, configured with advanced restrictions and 
security measures that limit their interaction with the rest of the infrastructure, thus minimizing the risks associated with their connectivity.